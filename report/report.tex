\documentclass[11pt,a4paper]{article}

% Packages
\usepackage[utf8]{inputenc}
\usepackage[T1]{fontenc}
\usepackage[english]{babel}
\usepackage{geometry}
\usepackage{setspace}
\usepackage{graphicx}
\usepackage{amsmath}
\usepackage{array}
\usepackage[colorlinks=true, linkcolor=black, urlcolor=blue, citecolor=black]{hyperref}
\usepackage{fancyhdr}
\usepackage{caption}
\usepackage{booktabs}
\usepackage{ragged2e}
\usepackage{enumitem}
\usepackage{amsmath, amssymb, booktabs}
\usepackage{siunitx}
\usepackage{multicol}
\usepackage{rotating}
\usepackage{tikz}
\usepackage{float}


% Arial-like font
\renewcommand{\familydefault}{\sfdefault}

% Page geometry
\geometry{
  a4paper,
  left=2.5cm,
  right=3.0cm,
  top=2.5cm,
  bottom=2.0cm
}

% Font and spacing
\onehalfspacing
\renewcommand{\familydefault}{\sfdefault}

% Title page
\begin{document}

\begin{titlepage}
\centering
% \vspace*{2cm}
  
\includegraphics[width=0.7\textwidth]{imgs/USB.png}\\[1cm]

\vspace{1.5cm}

\textbf{\Large University of South Bohemia} \\
\textbf{\large Faculty of Science} \\
\large Artificial Intelligence and Data Science, M.Sc. \\[2.5cm]

{\Large \textsc{ADHD Detection with EEG Signals}}\\[0.5cm]

Submitted for the course \textbf{Information Theory and Feature Engineering} \\
\textbf{Professor:} Doc. Ing. Ivo Bukovský, Ph.D. \\[3cm]

\begin{flushleft}
\begin{tabular}{ll}
Submitted by: & María Isabel Sánchez-O'Mullony Martínez \\
Student ID: & B24763 \\
Date: & \today \\
\end{tabular}
\end{flushleft}

\vspace{3cm}

\begin{flushright}
\begin{tabular}{ll}
Supervisor: & Doc. Ing. Ivo Bukovský, Ph.D. \\
\end{tabular}
\end{flushright}

\end{titlepage}


% ===============================
% Declaration
% ===============================
\section*{Declaration}
I declare that I have written this report by myself and have only used the sources and aids mentioned, and that I have 
marked direct and indirect citations as such. This report has not been submitted prior for any other examination.

I agree that the results of this study work / report may be used free of charge for research and lecturing purposes.

\newpage

% Table of Contents
\tableofcontents
\newpage

% List of Abbreviations
\section*{List of Abbreviations}
\begin{tabular}{ll}
    ADHD & Attention-Deficit/Hyperactivity Disorder \\
    EEG & Electroencephalography \\
    DC & Direct Current \\
\end{tabular}

\newpage

% Main Report Sections
% ===============================
% Introduction
% ===============================
\section{Introduction}\label{sec:Introduction}
Attention-Deficit/Hyperactivity Disorder (ADHD) is a prevalent neurodevelopmental condition that affects individuals across the lifespan. 
According to Furman~\cite{furman2005attention}, ADHD is not a single disease entity but rather a constellation of symptoms representing a 
final common behavioral pathway for a range of emotional, psychological, and learning difficulties. 

Although the behavioral manifestations of ADHD are well-documented, its neurophysiological underpinnings remain an active area of research. 
Electroencephalography (EEG) has been widely used to study ADHD, as it provides a non-invasive measure of brain electrical activity. 
Previous studies have identified characteristic EEG patterns in individuals with ADHD, particularly abnormalities in the frontal and central 
regions, often reflected as altered spectral power or atypical signal complexity. 

This study aims to investigate differences in EEG activity between healthy individuals and those diagnosed with ADHD, using the dataset 
provided by Sadeghi Bajestani *et al.*~\cite{dataset}. Specifically, the project focuses on (1) comparing the statistical and informational 
properties of EEG signals across both groups, (2) engineering features inspired by information theory—such as entropy and mutual information—to 
quantify signal complexity and connectivity, and (3) building a predictive model capable of distinguishing ADHD from healthy controls based on 
these features. This work bridges the domains of signal processing, feature engineering, and information theory to improve understanding and 
potential classification of ADHD-related brain activity.

% ===============================
% Data Description
% ===============================
\section{Data Description}\label{sec:Dataset}
The dataset~\cite{dataset} comprises EEG recordings from a total of 79 adult participants, including 42 healthy controls and 37 individuals 
diagnosed with ADHD (age range: 20-68 years; male/female ratio: 56/23). EEG signals were recorded under four experimental conditions: 
(1) resting state with eyes open, (2) resting state with eyes closed, (3) a cognitive challenge task, and (4) an auditory task involving 
listening to an ``omni harmonic'' sound stimulus. Recordings were obtained from five scalp locations—O1, F3, F4, Cz, and Fz—at a sampling rate 
of 256 Hz. These regions encompass occipital and frontal areas known to play key roles in attentional control and executive function.

\vspace{0.3cm}
\noindent\textbf{File organization and structure.}  
The data are provided as four MATLAB \texttt{.mat} files, corresponding to the experimental groups:
\begin{itemize}
    \item \texttt{FC.mat} - female control group  
    \item \texttt{MC.mat} - male control group  
    \item \texttt{FADHD.mat} - female ADHD group  
    \item \texttt{MADHD.mat} - male ADHD group
\end{itemize}

Each file contains a $1\times11$ cell array, where each cell represents one experimental task or condition. Within each cell, the data are 
stored as a three-dimensional matrix with dimensions \texttt{[subjects$\times$samples$\times$channels]}. For instance, a typical entry of size 
\texttt{13$\times$7680$\times$2} indicates 13 participants, 7680 time samples (corresponding to 30 seconds of EEG data at 256 Hz), and two 
recorded channels.

The specific configuration of each cell (i.e., channel pair and duration) is summarized as follows:

\begin{table}[h!]
\centering
\begin{tabular}{lllc}
\toprule
\textbf{Cell} & \textbf{Condition} & \textbf{Channels} & \textbf{Duration (s)} \\
\midrule
1 & Eyes open baseline & Cz, F4 & 30 \\
2 & Eyes closed & Cz, F4 & 20 \\
3 & Eyes open & Cz, F4 & 20 \\
4 & Cognitive challenge & Cz, F4 & 45 \\
5 & Pre-Omni harmonic baseline & Cz, F4 & 15 \\
6 & Omni harmonic assessment & Cz, F4 & 30 \\
7 & Eyes open baseline & O1, F4 & 30 \\
8 & Eyes closed & O1, F4 & 30 \\
9 & Eyes open & O1, F4 & 30 \\
10 & Eyes closed & F3, F4 & 45 \\
11 & Eyes closed & Fz, F4 & 45 \\
\bottomrule
\end{tabular}
\caption{Summary of EEG tasks, channel pairs, and recording durations.}
\label{tab:tasks}
\end{table}

One corrupted signal (subject 7 of the female ADHD group) was identified and excluded from further analysis. 

\vspace{0.3cm}
\noindent This hierarchical data organization allows for flexible analysis across multiple dimensions: by gender, diagnosis, condition, and 
channel pair. For subsequent preprocessing, each task will be reshaped into individual subject-channel recordings, ensuring consistent 
sampling durations across conditions.

% ===============================
% Data Visualization
% ===============================
\section{Data Visualization}\label{sec:visualization}
\subsection{Data Loading and Structure}
To analyze EEG signals from the ADHD dataset~\cite{dataset}, the recordings were imported from MATLAB \texttt{.mat} files using the 
\texttt{scipy.io.loadmat} library in Python. Four files were available, corresponding to each experimental group:
\texttt{FC.mat} (female control), \texttt{MC.mat} (male control), \texttt{FADHD.mat} (female ADHD), and \texttt{MADHD.mat} (male ADHD). 
Each file contained a $1\times11$ cell array, where each cell represented a specific experimental task and stored EEG data as a 
three-dimensional matrix with dimensions \texttt{[subjects x samples x channels]}.

A custom loading function (\texttt{load\_group}) was implemented to iterate through each cell, extract signal arrays, and 
store them in a structured \texttt{pandas DataFrame}. Each row in the resulting dataset corresponds to a single subject-task pair and includes 
the following metadata: participant group (Control/ADHD), gender, task number, subject identifier, and a two-channel EEG signal array. 
This organization facilitates subsequent preprocessing, visualization, and feature extraction. 
The corrupted EEG file corresponding to subject 7 of the female ADHD group was identified and excluded from analysis.

\subsection{Signal Visualization by Group and Task}

To inspect data quality and explore inter-subject variability, raw EEG signals were visualized for selected participants using 
\texttt{matplotlib}. Figure~\ref{fig:vis_task} presents the recordings from the first five female control subjects during Task~1 
(eyes-open baseline). Each subplot represents one subject, showing the two recorded channels.

\begin{figure}[H]
    \centering
    \includegraphics[width=1\textwidth]{imgs/vis_task.png}
    \caption{EEG signals from five female control participants during Task~1 (eyes-open baseline). Each subplot represents two simultaneously 
recorded channels.}\label{fig:vis_task}
\end{figure}

As illustrated in Figure~\ref{fig:vis_task}, the EEG traces display characteristic oscillatory patterns within an amplitude range 
of approximately $\pm200~\mu$V. Variations across participants are evident, with some showing strong low-frequency drifts or transient 
spikes, likely associated with ocular or muscular artifacts. Despite these variations, both channels exhibit similar temporal trends, 
suggesting functional coupling between the recorded electrode sites. These observations emphasize the need for further preprocessing steps, 
such as band-pass filtering (0.5-45~Hz), baseline correction, and normalization, to ensure consistent signal quality across participants.

\subsection{Comparison Between ADHD and Control Groups}
A complementary visualization was conducted to qualitatively compare EEG signals between ADHD and control subjects under the same task 
conditions. Figure~\ref{fig:vis_adhd} shows five pairs of male participants (Control vs.\ ADHD) performing Task~1, for Channel~0.

\begin{figure}[H]
    \centering
    \includegraphics[width=1\textwidth]{imgs/vis_adhd.png}
    \caption{Comparison of EEG signals between control and ADHD male participants during Task~1 for Channel~0. Each subplot shows a 
pair of subjects (Control vs.\ ADHD).}\label{fig:vis_adhd}
\end{figure}

In general, ADHD recordings exhibit greater amplitude fluctuations and reduced rhythmic stability compared to control subjects, who 
display smoother oscillatory activity. This increased irregularity suggests higher signal entropy and reduced neural synchronization in ADHD 
participants. Such qualitative patterns are consistent with prior research reporting atypical cortical activity and impaired oscillatory 
regulation in ADHD populations~\cite{lenartowicz2014use}. These visual differences motivate the extraction of quantitative features—
particularly information-theoretic measures such as entropy and mutual information—to characterize signal complexity and connectivity.

\subsection{Summary of Observations}
The exploratory visualization phase revealed several important insights:
\begin{itemize}
    \item EEG signals contain observable low-frequency drifts and high-amplitude artifacts, supporting the need for filtering and normalization.
    \item There is substantial inter-subject variability in signal amplitude and frequency content, even within the same condition.
    \item ADHD participants tend to exhibit higher signal variability and irregularity, indicating potential discriminative features for 
    subsequent classification.
\end{itemize}

These findings informed the design of the preprocessing and feature extraction pipeline, where statistical, spectral, and 
information-theoretic features were computed to enable quantitative group comparison and machine learning-based classification.

% \section{Preprocessing}\label{sec:preprocessing}
% EEG signals were detrended and band-pass filtered (0.5-45 Hz; 4th-order Butterworth). Where present, line noise was attenuated with a notch 
% filter at 50 Hz (harmonics were inspected via PSD and notched when required). Transient spikes were suppressed using a Hampel filter. Because 
% each task includes two electrodes, a bipolar derivation (ch1-ch2) was computed and used alongside the original channels. To standardize 
% duration across tasks, recordings were segmented into 5-s windows (50\% overlap); windows containing excessive amplitude (|x|>200 µV), 
% flatline, or near-zero variance were excluded. All subsequent feature extraction and model fitting were performed with subject-wise 
% cross-validation to prevent information leakage; scalers were fit on training folds and applied to test folds.

% ===============================
% Preprocessing
% ===============================
\section{Preprocessing}\label{sec:preprocessing2}
EEG data are highly sensitive to non-neural artifacts arising from muscle activity, eye blinks, and small head or body movements. 
Because the dataset used in this study was recorded non-invasively from scalp electrodes, such artifacts are inevitable and must be 
carefully mitigated to preserve the validity of the neural information~\cite{chaddad2023electroencephalography}. The objective of the 
preprocessing stage was therefore to reduce non-cerebral noise and standardize the signals prior to feature extraction.

\subsection{Detrending}
EEG recordings were detrended by subtracting the mean amplitude of each channel to remove Direct Current (DC) offsets and slow baseline 
drifts~\cite{de2018robust}. EEG recordings often contain a small DC offset — a constant voltage bias caused by electrode-skin impedance 
and amplifier drift. This offset shifts the entire signal above or below zero microvolts without carrying any neural information. This 
step prevents artificial low-frequency power and ensures that subsequent analyses reflect genuine neural oscillations. Although the 
visual appearance of the waveforms remains largely unchanged, detrending is essential to prevent bias in later spectral and entropy-based 
feature extraction.

To verify the effectiveness of the detrending procedure, the mean amplitude of each EEG channel was computed before and after baseline 
correction. As shown below in~\ref{tab:detrend_compare}, the mean values were reduced to approximately zero following detrending, 
confirming successful removal of the DC offset while preserving the oscillatory structure of the signal.

\begin{table}[h!]
\centering
\begin{tabular}{lcc}
\toprule
 & \textbf{Channel 1} & \textbf{Channel 2} \\
\midrule
Mean before detrending (\textmu V) & 1.5269 & 0.0936 \\
Mean after detrending (\textmu V) & -1.74$\times$10$^{-15}$ & 1.18$\times$10$^{-16}$ \\
\bottomrule
\end{tabular}
\caption{Verification of mean amplitude before and after detrending.}
\label{tab:detrend_compare}
\end{table}

The results confirm that the detrending operation correctly centered the EEG signals around zero microvolts, thereby eliminating slow 
baseline drifts without altering the underlying temporal dynamics. This ensures that subsequent frequency and entropy analyses are not 
biased by non-neural voltage offsets.


% \subsection{Signal Filtering and Cleaning}
% Each recording was first detrended by removing its mean to eliminate slow DC components. A fourth-order Butterworth band-pass filter 
% between 0.5~Hz and 45~Hz was then applied to suppress slow drifts and high-frequency muscular or environmental noise while preserving 
% the main EEG frequency bands (delta, theta, alpha, beta, and low-gamma). In addition, a notch filter at 50~Hz was used to attenuate 
% power-line interference. 

% Transient spikes were suppressed using a Hampel filter, and any segment exceeding an absolute amplitude of $\pm$200~\textmu V or showing 
% near-zero variance was excluded. These steps effectively reduced the influence of eye blinks and abrupt movements while maintaining 
% the oscillatory structure of the neural signal. Figure~\ref{fig:filtering_example} illustrates an example EEG trace before and after 
% filtering, demonstrating the removal of low-frequency drifts and high-frequency noise.

% % \begin{figure}[h!]
% % \centering
% % \includegraphics[width=0.8\textwidth]{example_filtering.png}
% % \caption{Example EEG trace before (top) and after (bottom) preprocessing. Band-pass and notch filtering remove slow drifts and 
% % high-frequency noise while preserving neural oscillatory content.}
% % \label{fig:filtering_example}
% % \end{figure}

% \subsection{Segmentation and Standardization}
% Recordings varied in duration from 15~s to 45~s depending on the task. To ensure temporal consistency, each signal was divided into 
% fixed 5~s windows (1280~samples at 256~Hz) with 50\% overlap. Windows containing artifacts according to the amplitude or variance 
% criteria were automatically rejected. Each remaining segment was standardized using z-score normalization to ensure comparability 
% across participants and tasks. 

% A derived bipolar channel (difference between the two electrodes) was also computed and included alongside the original channels to 
% enhance sensitivity to local cortical activity while suppressing common reference noise.

% \subsection{Rationale and Validity Considerations}
% Although advanced artifact removal techniques such as independent component analysis (ICA) are commonly used in high-density EEG, 
% the present dataset provides only two channels per task, which limits the use of such methods. The adopted preprocessing strategy 
% therefore focused on robust signal cleaning using filtering, threshold-based rejection, and normalization. These measures substantially 
% improve signal quality while minimizing the risk of removing genuine neural features potentially related to ADHD.

% The cleaned and standardized signals resulting from this pipeline form the basis for subsequent feature extraction, where both 
% statistical and information-theoretic descriptors are computed to characterize EEG dynamics across groups.

% \section{Discussion and Limitations}\label{sec:discussion}

% One of the major challenges in EEG-based analysis lies in distinguishing true neural activity from non-neural artifacts. 
% Because EEG is recorded from scalp electrodes rather than intracranial sources, the signals are inherently vulnerable to contamination 
% by muscular, ocular, and movement-related activity. While the preprocessing pipeline used in this study—comprising band-pass and notch 
% filtering, spike suppression, and amplitude-based rejection—successfully mitigated much of this interference, complete artifact removal 
% is not possible with only two recording channels. Consequently, some residual artifacts may persist and potentially influence the extracted 
% features.

% This limitation affects the validity of any EEG-based classification, as a portion of the observed differences between ADHD and control 
% participants could reflect artifact-related variability rather than genuine neurophysiological differences. Nevertheless, the consistency 
% of the cleaned signals across subjects and the preservation of characteristic oscillatory patterns (particularly within alpha and beta bands) 
% support the interpretation that the derived features primarily capture neural dynamics. Future work using high-density EEG, simultaneous 
% eye-tracking or EMG recordings, and advanced source-separation methods (e.g., ICA or wavelet-based artifact correction) would allow for a 
% more precise dissociation between artifacts and ADHD-related brain activity.

\newpage

\cleardoublepage
\addcontentsline{toc}{section}{References}
\bibliographystyle{IEEEtran}
\bibliography{references}

\newpage

% AI Prompt Used
% \section*{AI Prompt Used}
% \begin{quote}
% 1. What should I do

% \end{quote}

\end{document}
