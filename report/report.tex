\documentclass[11pt,a4paper]{article}

% Packages
\usepackage[utf8]{inputenc}
\usepackage[T1]{fontenc}
\usepackage[english]{babel}
\usepackage{geometry}
\usepackage{setspace}
\usepackage{graphicx}
\usepackage{amsmath}
\usepackage{array}
\usepackage[colorlinks=true, linkcolor=black, urlcolor=blue, citecolor=black]{hyperref}
\usepackage{fancyhdr}
\usepackage{caption}
\usepackage{booktabs}
\usepackage{ragged2e}
\usepackage{enumitem}
\usepackage{amsmath, amssymb, booktabs}
\usepackage{siunitx}
\usepackage{multicol}
\usepackage{rotating}
\usepackage{tikz}
\usepackage{float}


% Arial-like font
\renewcommand{\familydefault}{\sfdefault}

% Page geometry
\geometry{
  a4paper,
  left=2.5cm,
  right=3.0cm,
  top=2.5cm,
  bottom=2.0cm
}

% Font and spacing
\onehalfspacing
\renewcommand{\familydefault}{\sfdefault}

% Title page
\begin{document}

\begin{titlepage}
\centering
% \vspace*{2cm}
  
\includegraphics[width=0.7\textwidth]{imgs/USB.png}\\[1cm]

\vspace{1.5cm}

\textbf{\Large University of South Bohemia} \\
\textbf{\large Faculty of Science} \\
\large Artificial Intelligence and Data Science, M.Sc. \\[2.5cm]

{\Large \textsc{ADHD Detection with EEG Signals}}\\[0.5cm]

Submitted for the course \textbf{Information Theory and Feature Engineering} \\
\textbf{Professor:} Doc. Ing. Ivo Bukovský, Ph.D. \\[3cm]

\begin{flushleft}
\begin{tabular}{ll}
Submitted by: & María Isabel Sánchez-O'Mullony Martínez \\
Student ID: & B24763 \\
Date: & \today \\
\end{tabular}
\end{flushleft}

\vspace{3cm}

\begin{flushright}
\begin{tabular}{ll}
Supervisor: & Doc. Ing. Ivo Bukovský, Ph.D. \\
\end{tabular}
\end{flushright}

\end{titlepage}


% Declaration
\section*{Declaration}
I declare that I have written this report by myself and have only used the sources and aids mentioned, and that I have 
marked direct and indirect citations as such. This report has not been submitted prior for any other examination.

I agree that the results of this study work / report may be used free of charge for research and lecturing purposes.

\newpage

% Table of Contents
\tableofcontents
\newpage

% List of Abbreviations
\section*{List of Abbreviations}
\begin{tabular}{ll}
    ADAS & Advanced Driver Assistance Systems \\
    ADC & Analog-to-Digital Converter \\
    AMCW & Amplitude Modulated Continuous Wave \\
    CAN & Controller Area Network \\
\end{tabular}

\newpage

% Main Report Sections
\section{Introduction}\label{sec:Introduction}
Attention-Deficit/Hyperactivity Disorder (ADHD) is a prevalent neurodevelopmental condition that affects individuals across the lifespan. 
According to Furman \cite{furman2005attention}, ADHD is not a single disease entity but rather a constellation of symptoms representing a 
final common behavioral pathway for a range of emotional, psychological, and learning difficulties. 

Although the behavioral manifestations of ADHD are well-documented, its neurophysiological underpinnings remain an active area of research. 
Electroencephalography (EEG) has been widely used to study ADHD, as it provides a non-invasive measure of brain electrical activity. 
Previous studies have identified characteristic EEG patterns in individuals with ADHD, particularly abnormalities in the frontal and central 
regions, often reflected as altered spectral power or atypical signal complexity. 

This study aims to investigate differences in EEG activity between healthy individuals and those diagnosed with ADHD, using the dataset 
provided by Sadeghi Bajestani *et al.* \cite{dataset}. Specifically, the project focuses on (1) comparing the statistical and informational 
properties of EEG signals across both groups, (2) engineering features inspired by information theory—such as entropy and mutual information—to 
quantify signal complexity and connectivity, and (3) building a predictive model capable of distinguishing ADHD from healthy controls based on 
these features. This work bridges the domains of signal processing, feature engineering, and information theory to improve understanding and 
potential classification of ADHD-related brain activity.

\section{Data Description}\label{sec:Dataset}
The dataset \cite{dataset} comprises EEG recordings from a total of 79 adult participants, including 42 healthy controls and 37 individuals 
diagnosed with ADHD (age range: 20-68 years; male/female ratio: 56/23). EEG signals were recorded under four experimental conditions: 
(1) resting state with eyes open, (2) resting state with eyes closed, (3) a cognitive challenge task, and (4) an auditory task involving 
listening to an ``omni harmonic'' sound stimulus. Recordings were obtained from five scalp locations—O1, F3, F4, Cz, and Fz—at a sampling rate 
of 256 Hz. These regions encompass occipital and frontal areas known to play key roles in attentional control and executive function.

\vspace{0.3cm}
\noindent\textbf{File organization and structure.}  
The data are provided as four MATLAB \texttt{.mat} files, corresponding to the experimental groups:
\begin{itemize}
    \item \texttt{FC.mat} - female control group  
    \item \texttt{MC.mat} - male control group  
    \item \texttt{FADHD.mat} - female ADHD group  
    \item \texttt{MADHD.mat} - male ADHD group
\end{itemize}

Each file contains a $1\times11$ cell array, where each cell represents one experimental task or condition. Within each cell, the data are 
stored as a three-dimensional matrix with dimensions \texttt{[subjects x samples x channels]}. For instance, a typical entry of size 
\texttt{13x7680x2} indicates 13 participants, 7680 time samples (corresponding to 30 seconds of EEG data at 256 Hz), and two recorded channels.

The specific configuration of each cell (i.e., channel pair and duration) is summarized as follows:

\begin{table}[h!]
\centering
\begin{tabular}{lllc}
\toprule
\textbf{Cell} & \textbf{Condition} & \textbf{Channels} & \textbf{Duration (s)} \\
\midrule
1 & Eyes open baseline & Cz, F4 & 30 \\
2 & Eyes closed & Cz, F4 & 20 \\
3 & Eyes open & Cz, F4 & 20 \\
4 & Cognitive challenge & Cz, F4 & 45 \\
5 & Pre-Omni harmonic baseline & Cz, F4 & 15 \\
6 & Omni harmonic assessment & Cz, F4 & 30 \\
7 & Eyes open baseline & O1, F4 & 30 \\
8 & Eyes closed & O1, F4 & 30 \\
9 & Eyes open & O1, F4 & 30 \\
10 & Eyes closed & F3, F4 & 45 \\
11 & Eyes closed & Fz, F4 & 45 \\
\bottomrule
\end{tabular}
\caption{Summary of EEG tasks, channel pairs, and recording durations.}
\label{tab:tasks}
\end{table}

One corrupted signal (subject 7 of the female ADHD group) was identified and excluded from further analysis. 

\vspace{0.3cm}
\noindent This hierarchical data organization allows for flexible analysis across multiple dimensions: by gender, diagnosis, condition, and 
channel pair. For subsequent preprocessing, each task will be reshaped into individual subject-channel recordings, ensuring consistent 
sampling durations across conditions.

\section{Preprocessing}



% \subsection{An image}
% \label{sec:ToF}
% \textbf{A subsection}

% \begin{figure}[H]
%     \centering
%     \includegraphics[width=1\textwidth]{imgs/USB.png}
%     \caption{USB logo~\cite{johansson2005vehicle}}\label{fig:measurementsprinciples}
% \end{figure}


\newpage

\cleardoublepage
\addcontentsline{toc}{section}{References}
\bibliographystyle{IEEEtran}
\bibliography{references}

\newpage

% AI Prompt Used
\section*{AI Prompt Used}
\begin{quote}
1. What should I do

\end{quote}

\end{document}
